\newcommand{\hamburger}{%
  \vbox{%
    \hrule width 1em height 1pt\smallskip
    \hrule width 1em height 1pt\smallskip
    \hrule width 1em height 1pt}%
}
\chapter{RSRM Operations}
\gls{rsrm} simplifies identifying rollingstock by associating \gls{rfid} tags with specific road names and numbers.
When a connected reader reads an \gls{rfid} tag, \gls{rsrm} searches its database for a matching entry. If a match is found, the application displays the corresponding 
road name and number of the rollingstock. If no match is found, the application provides input fields for the user to manually enter this information, 
creating a new association in the database.

\section{About Page}

The intial page when the application is opened in a web browser provides:
\begin{itemize}
    \item A version number of the application;
    \item A concise overview of the application's purpose and technology stack;
    \item The current database statistics indicating:
    \begin{itemize}
        \item The number of rollingstock entries in the database and how many have \gls{rfid} tags; and
        \item The number of micro-controllers in the database and how many are RFID readers.
    \end{itemize}
\end{itemize}
Figure \ref{fig:about1} shows the "About" page of the RAILS RS RFID Manager application.

\begin{figure}[H]
    \centering
    \includegraphics[scale=0.33]{./images/about.png}
    \caption{About Page}
    \label{fig:about1}
\end{figure}

The hamburger menu icon \hamburger, to the left of "RAILS" represents a hidden navigation menu that can be toggled open and closed. 
When the icon is clicked or tapped on the menu for navigation to different pages of this application is revealed. This menu  
contains the links: "Reader," "Admin," "Rollingstock," "AAR Codes," "About" (this page). 
Figure \ref{fig:about2} shows the "About" page with the navigation menu opened.

\begin{figure}[H]
    \centering
    \includegraphics[scale=0.33]{./images/about-sidemenu.png}
    \caption{Sidemenu}
    \label{fig:about2}
\end{figure}

\section{Reader Page}
This page displays a log of recently read \gls{rfid} tags and their associated information. It provides real-time or near real-time 
tracking of rolling stock as they pass over \gls{rfid} readers. To access the "RFID Tags Read" page from the main navigation menu on the 
left side of the screen, click on the "Reader" link. The "RFID Tags Read" page will be displayed.\\
\\
Figure \ref{fig:reader1} shows the initial "RFID Tags Read" page of the \gls{rsrm} application. Table displayed columns are:
\begin{itemize}
    \item Time: The date and time the \gls{rfid} tag was read;
    \item Sensor: The name of the micro-controller the \gls{rfid} reader is connected to;
    \item Reader: The identifier of the \gls{rfid} reader that detected the tag;
    \item Road Name and Number: This column has three possible entries:
    \begin{itemize} 
        \item "No data avaibale" indicates that no \gls{rfid} reader has sent any data to the \gls{rsrm} application;
        \item The road name and number of the rollingstock associated with the read \gls{rfid} tag.
        \item "Not registered" indicates that the \gls{rfid} tag was not found in the database.
    \end{itemize}
    \item AAR: The AAR code represents the type of rollingstock;
    \item Color: The color of the rollingstock;
    \item Register: If the \gls{rfid} tag is not in the database an icon appears that the user may click to add the associated 
    rollingstock to the database.
\end{itemize}
In the top left corner below the title is an \gls{rfid} icon that indicates the status of the \gls{rsrm} applications's connection to 
the \gls{mqtt} broker, green shows a good connection and red a failed connection.\\
The "CLEAR" button reloads the page and clears the log of read \gls{rfid} tags.
\begin{figure}[H]
    \centering
    \includegraphics[scale=0.33]{./images/reader.png}
    \caption{Initial RFID Tags Read Page}
    \label{fig:reader1}
\end{figure}
Figure \ref{fig:reader2} displays a log of with actual data read RFID tags and their associated information. The user can see that 
some tags are associated with known rollingstock ("CN 521491," "SCOX 1408," "CN 114520"), while others are marked as "Not Registered." 
This indicates that these tags are not yet linked to any rolling stock in the system's database. This is important for identifying 
new or unknown rollingstock. The user can click on the "Register" icon to add the rollingstock to the database.

\begin{figure}[H]
    \centering
    \includegraphics[scale=0.33]{./images/reader1.png}
    \caption{RFID Tags Read}
    \label{fig:reader2}
\end{figure}

When an RFID tag is read by a reader but is not found in the system's database, the "RFID Tags Read" section will display 
"Not Registered" in the "Road Name and Number" column. To associate this tag with a piece of rolling stock, the user may wish 
to register it. To do so, the user can click on the "Register" icon in the "Register" column. This action will open a dialog 
box where the user can enter the necessary information to create a new association between the RFID tag and the rolling stock. 
Once the information is saved, the new entry will be added to the database, and the "Not Registered" status will be updated to 
reflect the newly registered rolling stock.\\
\\
Figure \ref{fig:reader3} displays the "Rollingstock Registration" dialog box. It allows the user to associate the unregistered 
\gls{rfid} to a rollingstock to the system. The dialog box has:
\begin{itemize}
    \item Road Name: Enter the name of the railroad company or owner of the rollingstock;
    \item Road Number: Enter the unique identifying number assigned to the specific piece of rollingstock;
    \item two buttons:
    \begin{itemize}
        \item CANCEL: Closes the dialog box without saving the association to a rollingstock; and
        \item SAVE: Adds the \gls{rfid} tag to the identified rollingstock.
    \end{itemize}
\end{itemize}
\begin{figure}[H]
    \centering
    \includegraphics[scale=0.33]{./images/reader2.png}
    \caption{Revised RFID Tags Read}
    \label{fig:reader3}
\end{figure}
When the user clicks on the "SAVE" button, the dialog box will close, and adds the \gls{rfid} tag to the identified rollingstock 
and displayed in the "RFID Tags Read" page as shown in Figure \ref{fig:reader4}.
\begin{figure}[H]
    \centering
    \includegraphics[scale=0.33]{./images/reader3.png}
    \caption{Rollingstock Registration}
    \label{fig:reader4}
\end{figure}
If the user enters a road name and number that does not exist in the system's database, the system will expand the dialog box to allow 
the user to create a new rollingstock entry. The expanded dialog box, as shown in figure \ref{fig:reader6} has additional fields for the user to enter:
\begin{itemize}
    \item AAR: Select the AAR code corresponding to the rollingstock type from a dropdown list;
    \item Color: Enter the color of the rollingstock;
    \item along with the two buttons:
    \begin{itemize}
        \item CANCEL: Closes the dialog box without saving the new rollingstock; and
        \item SAVE: Saves the information entered and adds the new rollingstock to the database.
    \end{itemize}
\end{itemize}
\begin{figure}[H]
    \centering
    \includegraphics[scale=0.33]{./images/reader4.png}
    \caption{Expanded Rollingstock Registration}
    \label{fig:reader5}
\end{figure}
When the user clicks on the "SAVE" button, the dialog box will close, and adds the new rollingstock to the database
and displayed in the "RFID Tags Read" page as shown in Figure \ref{fig:reader6}.
\begin{figure}[H]
    \centering
    \includegraphics[scale=0.33]{./images/reader5.png}
    \caption{RFID Tags Read}
    \label{fig:reader6}
\end{figure}

\section{Admin Page}
This page allows the user to manage the RFID readers (micro-controllers) connected to the system. To access the "Admin RFID Micros" page 
from the main navigation menu on the left side of the screen, click on the "Admin" link. The "Admin RFID Micros" page will be displayed. 
While the focus of the \gls{rsrm} "Admin RFID Micros" page is on managing \gls{rfid} micro-controllers any micro-controller can be managed.
Figure \ref{fig:admin1} shows the "Admin RFID Micros" page of the \gls{rsrm} application. Table displayed columns are:
\begin{itemize}
    \item Name: The name of the micro-controller (e.g., RfidRdr00, ToCntlr01, RfidRdr03);
    \item IP Address: The IP address of the micro-controller;
    \item Date Time: The last recorded communication both data and heartbeat messages time with the micro-controller;
    \item Location: The physical location on the layout of the micro-controller sensor (\gls{rfid} reader);
    \item Status: An indicator of the reader's current status, represented by a color-coded icon:
    \begin{itemize}
        \item Green: Reader is active and communicating correctly.
        \item Yellow: Reader is connected but may have issues or hasn't communicated recently.
        \item Red: Reader is not communicating or is offline.
    \end{itemize}
    \item Actions: This column contains icons, for editing and deleting an existing micro-controller:
    \begin{itemize}
        \item Clicking the "Edit" (pencil) icon will open a dialog box allowing the user to modify the micro-controller's information; and
        \item Clicking the "Delete" (trash can) icon will remove the micro-controller entry from the system. A confirmation dialog box will 
        appear to prevent accidental deletions.
    \end{itemize}
\end{itemize}

Clicking on any of the table headers, except the "Actions" header, sorts the list by that header alphanumerically, either forward or reverse. For example 
if the user clicked on the "IP" header the entire list of micro-controllers would be sorted by their IP attribute. An arrow appears to the right of the 
header indicating which diretion the sort occurs.\\
\\
This table is paginated, showing the group of rollingstocks displayed (e.g. "1-5") and the total number of rollingstocks (e.g. "5") conatined in the database, along 
with navigation buttons ("<" and ">") to move between pages. The user can also select the number of items to display per page, where the default setting is 10.
The "ADD MICRO" button allows the user to add new rollingstock to the system.\\
\\
The "EXPORT RFID RS" Button exports the rollingstock data containing all of the rollingstock attributes to a CSV file.\\
\\
The "PRINT RFID REPORT" Button generates a printable \gls{pdf} report of the rollingstock data containing \gls{rfid}, Road Name, Road Number, Color, \gls{aar} Code, 
and Description of the rollingstock attributes.

\begin{figure}[H]
    \centering
    \includegraphics[scale=0.33]{./images/admin.png}
    \caption{Admin Page}
    \label{fig:admin1}
\end{figure}

Figure \ref{fig:admin2} shows a dialog box, which is accessed by clicking the "ADD MICRO" button in the "Admin RFID Micros" page. It allows the user to add 
a new entry for rollingstock to the system. The dialog box has:
\begin{itemize}
    \item Name: Enter the name of the micro-controller (e.g., RfidRdr00, ToCntlr01, RfidRdr03);
    \item IP Address: Enter the IP address of the micro-controller;
    \item Time: Enter a time in \gls{epoch} format;
    \item Purpose: Enter the purpose of the micro-controller (e.g., RFID Reader, Turnout Controller, RFID Reader);
    \item Location: Enter the physical location on the layout of the micro-controller sensor (e.g., \gls{rfid} reader);
    \item Status: Enter the micro-controller's current status (e.g., Up, Problem, Down).
\end{itemize}
The dialog box also has two buttons:
\begin{itemize}
    \item CANCEL: Closes the dialog box without saving the new micro-controller; and
    \item SAVE: Saves the information entered and adds the new micro-controller to the database and the new micro-controller will appear in the list.
\end{itemize}

\begin{figure}[H]
    \centering
    \includegraphics[scale=0.33]{./images/admin-micro-add.png}
    \caption{Admin Page}
    \label{fig:admin2}
\end{figure}

Figure \ref{fig:admin3} shows the "Micro" edit dialog box. It displays the details of an existing a micro-controller that 
the user clicked or tapped the pencil icon on. This dialog box has the same fields and buttons as the "New Micro" dialog box, 
but the SAVE button modifies the existing record in the database instead of creating a new one.

\begin{figure}[H]
    \centering
    \includegraphics[scale=0.33]{./images/admin-micro-edit.png}
    \caption{Admin Page}
    \label{fig:admin3}
\end{figure}

\section{Rollingstock Page}
This page provides a searchable and manageable list of all rollingstocks tracked by the system. To access the "Rollingstock" page 
from the main navigation menu on the left side of the screen, click on the "Rollingstock" link. The "Inventory of Rollingstock" view will be displayed.
Figure \ref{fig:rollingstock1} shows the "Rollingstock" page of the \gls{rsrm} application. Table displayed columns are:
\begin{itemize}
    \item Road Name: The name of the railroad company or owner of the rollingstock (e.g., ACFX, ADCX, ALNX, ALPX).
    \item Road Number: The unique identifying number assigned to the specific piece of rollingstock (e.g., 44763, 5425, 396029).
    \item \gls{aar}: The AAR code represents the type of rollingstock (e.g., LO, RM).
    \item Color: The color of the rollingstock (e.g., Grey, White, Blue).
    \item \gls{rfid}: The unique identifier of the \gls{rfid} tag associated with the rollingstock.
    \item Actions: This column contains icons, for editing and deleting an existing AAR code:
    \begin{itemize}
        \item Clicking the "Edit" (pencil) icon will open a dialog box allowing the user to modify the rollingstock's information; and
        \item Clicking the "Delete" (trash can) icon will remove the rollingstock entry from the system. A confirmation dialog box will 
        appear to prevent accidental deletions.
    \end{itemize}
\end{itemize}

A search bar (labeled with a magnifying glass icon and "Search") allows the user to filter the list of rollingstock by various strings (regradless of case). 
The table will automatically filter to display only entries that match your search. The search is performed across all displayed columns (Road Name, Road 
Number, AAR, Color, RFID). The following are some search examples:
\begin{itemize}
    \item Searching for "ACFX" or "acfx" will display all rollingstock with the Road Name "ACFX";
    \item Searching for "44" will display the rollingstock containing the Road Number "44" including "44763";
    \item Searching for "Blue" will display all blue rollingstock.
\end{itemize}
    
Clicking on any of the table headers, except the "Actions" header, sorts the list by that header alphanumerically, either forward or reverse. For example if 
the user clicked on the "Color" header the entire list of rollingstocks would be sorted by their color attribute. An arrow appears to the right of the 
header indicating which diretion the sort occurs.\\
\\
This table is paginated, showing the group of rollingstocks displayed (e.g. "1-10") and the total number of rollingstocks (e.g. "211") contained in the database, along 
with navigation buttons ("<" and ">") to move between pages. The user can also select the number of items to display per page, where the default setting is 10.
The "NEW ROLLINGSTOCK" button allows the user to add new rollingstock to the system.

\begin{figure}[H]
    \centering
    \includegraphics[scale=0.33]{./images/rs.png}
    \caption{Rollingstock Page}
    \label{fig:rollingstock1}
\end{figure}

Figure \ref{fig:rollingstock2} shows a dialog box, which is accessed by clicking the "NEW ROLLINGSTOCK" button in the "Inventory of Rollingstock" page. It allows the user to add 
a new entry for rollingstock to the system. The dialog box has:
\begin{itemize}
    \item twenty four input fields;
    \begin{itemize}
        \item Road Name (required): Enter the name of the railroad company or owner of the rollingstock;
        \item Road Number (required): Enter the unique identifying number assigned to the specific piece of rollingstock;
        \item AAR (required): Enter the AAR code corresponding to the rollingstock type;
        \item Color: Enter the color of the rollingstock (e.g., "Grey," "Blue," "Red");
        \item Description: Enter any additional descriptive information about the rollingstock;
        \item Number Built: Enter the total number of units built in this series;
        \item Builder: Enter the name of the manufacturer who built the rollingstock;
        \item Built Date: Enter the date the rollingstock was built;
        \item In Service Date: Enter the date the rollingstock was put into service;
        \item Inside Length, Inside Height, Inside Width, Lt Weight: Enter these fields as they are are specific to certain types of rollingstock. They are used to record relevant dimensions and weight information.
        \item Load Limit: Enter the maximum weight capacity of the rollingstock;
        \item Load Types: Enter the types of cargo the rollingstock is designed to carry (e.g., "General Merchandise," "Coal," "Automobiles");
        \item Capacity: Enter the rollingstock's specific capacity (e.g., "100 tons," "50 cars," etc.);
        \item Home Location: Enter where the rollingstock is typically based or maintained;
        \item Last Maintenance: Enter the date of the last maintenance performed on the rollingstock;
        \item Status: Enter the rollingstock's current operational status (e.g., "In Service," "Out of Service," or "Maintenance”);
        \item RFID Tag: Enter the unique identifier of the RFID tag associated with the rollingstock;
        \item Weight: Enter the weight of the model rollingstock;
        \item Length: Enter the length of the model rollingstock;
        \item Image ID: Enter an identifier for an associated image of the rollingstock; and
        \item Notes: Enter any additional notes or observations about the rollingstock.
    \end{itemize}
    \item two buttons:
    \begin{itemize}
        \item CANCEL: Closes the dialog box without saving the new rollingstock; and
        \item SAVE: Saves the information entered and adds the new rollingstock to the database and the new code will appear in the rollingstock list.
    \end{itemize}
\end{itemize} 
\begin{figure}[H]
    \centering
    \includegraphics[scale=0.33]{./images/rs-add.png}
    \caption{Rolling Stock Page}
    \label{fig:rollingstock2}
\end{figure}

Figure \ref{fig:rollingstock3} shows the "Rollingstock" edit dialog box. It displays the details of an existing a rollingstock that 
the user clicked or tapped the pencil icon on. This dialog box has the same fields and buttons as the "New Rolling Stock" dialog box, 
but the SAVE button modifies the existing record in the database instead of creating a new one.

\begin{figure}[H]
    \centering
    \includegraphics[scale=0.33]{./images/rs-edit.png}
    \caption{Rolling Stock Page}
    \label{fig:rollingstock3}
\end{figure}

\section{AAR Code Page}
This page provides a manageable list of \gls{aar} car codes, standardized codes used to classify different types of railroad cars, allowing users to understand 
the different types of railroad cars the system tracks. To access the "AAR Code" page from the main navigation menu on the left side of the screen, click on 
the "AAR Codes" link. Then "AAR Codes" page will be displayed.\vspace{5mm} \\
\\
Figure \ref{fig:aar1} shows the "AAR Codes" page of the \gls{rsrm} application. Table displayed columns are:
\begin{itemize}
    \item \gls{aar}: The \gls{aar} code (e.g., BE, BMR, DA).
    \item \gls{rs} Type: The general type of rollingstock (e.g., Passenger, Freight, Locomotive).
    \item Description: A detailed description of the car type.
    \item Actions: This column contains icons, for editing and deleting an existing AAR code:
    \begin{itemize}
        \item Clicking the "Edit" (pencil) icon will open a dialog box allowing the user to modify the \gls{aar} Code's information; and
        \item Clicking the "Delete" (trash can) icon will remove the \gls{aar} code entry from the system. A confirmation dialog box will 
        appear to prevent accidental deletions.
    \end{itemize}
\end{itemize}

Clicking on any of the table headers, except the "Actions" header, sorts the list by that header alphanumerically, either forward or reverse. For example 
if the user clicked on the "RS Type" header the entire list of \gls{aar} codes would be sorted by their \gls{rs} type attribute. An arrow appears 
to the right of the header indicating which diretion the sort occurs. \\
\\
This table is paginated, showing the group of AAR Codes displayed (e.g. "1-10") and the total number of AAR Codes (e.g. "34") conatined in the database, along 
with navigation buttons ("<" and ">") to move between pages. The user can also select the number of items to display per page, where the default setting is 10.
The "ADD AAR CODE" button allows the user to add new \gls{aar} codes to the system.
   
\begin{figure}[H]
    \centering
    \includegraphics[scale=0.33]{./images/aar.png}
    \caption{AAR Code Page}
    \label{fig:aar1}
\end{figure}

Figure \ref{fig:aar2} shows a dialog box, which is accessed by clicking the "ADD AAR CODE" button in the "AAR Codes" page. It allows the user to add 
a new entry for \gls{aar} Code to the system. 
The dialog box has:
\begin{itemize}
    \item three input fields, \gls{aar} Code, RS Type, and Description;
    \item two buttons:
    \begin{itemize}
        \item CANCEL: Closes the dialog box without saving the new \gls{aar} code; and
        \item SAVE: Saves the information entered and adds the new \gls{aar} code to the database and the new code will appear in the \gls{aar} Codes list.
    \end{itemize}
\end{itemize} 

\begin{figure}[H]
    \centering
    \includegraphics[scale=0.33]{./images/aar-add.png}
    \caption{Add AAR Code Dialog Box}
    \label{fig:aar2}
\end{figure}

Figure \ref{fig:aar3} shows the "AAR Code" edit dialog box. It displays the details of an existing \gls{aar} code that 
the user clicked or tapped the pencil icon on. This dialog box has the same fields and buttons as the "New AAR Code" dialog box, 
but the SAVE button modifies the existing record in the database instead of creating a new one.

\begin{figure}[H]
    \centering
    \includegraphics[scale=0.33]{./images/aar-edit.png}
    \caption{Edit AAR Code Dialog Box}
    \label{fig:aar3}
\end{figure}
