\chapter{Docker Set Up}

\section{Docker Installation}
\label{sec:dockerinstall}
The \gls{rails} \glspl{spa} are implemented as Docker containers. The Docker containers are built and pushed to Docker Hub. The Docker containers are then pulled from Docker Hub and run on the host machine. 
The host machine must have Docker installed. Docker is a platform for developing, shipping, and running applications in containers. Docker can be installed on Windows, macOS, and Linux. 
The installation instructions for Docker can be found at \href{https://docs.docker.com/get-docker/}{https://docs.docker.com/get-docker/}.\\
\\
To see additioanl information on Docker installation of a complete \gls{rails} see \href{https://github.com/djbristow/RAILS/blob/master/Documentation/rails-docker.pdf}{Docker Implementation}.\\
\section{Broker Configuration}
\label{sec:brokerconfig}
The \gls{rails} \glspl{spa} use the Mosquitto broker to communicate with each other. The Mosquitto broker is an open-source message broker that implements the \gls{mqtt} protocol. The Mosquitto broker 
must be configured to allow the \gls{rails} \glspl{spa} to communicate with each other. The Mosquitto broker configuration file must be edited to allow the \gls{rails} \glspl{spa} to communicate with 
each other. The configuration file is found in the virtual storage whose volume name is “mosquitto”. To establish and modify the configuration file the following steps are taken:
\begin{enumerate}
    \item Create a Docker volume for the Mosquitto broker.
    \begin{verbatim}
    docker volume create --name mosquitto
    \end{verbatim}
    \item Run a Docker container for the first time the Mosquitto broker, which will create the initial configuration file.
    \begin{verbatim}
    docker run -it --name myMqttBrkr -p 1883:1883 -p 9001:9001 --rm
    -v mosquitto:/mosquitto -d eclipse-mosquitto
    \end{verbatim}
    \item Stop the broker container.
    \begin{verbatim}    
    docker stop myMqttBrkr
    \end{verbatim}
    \item Locate the configuration file in the "mosquitto" volume.
    \begin{verbatim}    
    docker inspect mosquitto
    \end{verbatim}
    \item Edit the configuration file modifying or adding the following lines:
    \begin{verbatim}
    listener 1883
    allow_anonymous true
    socket_domain ipv4
    \end{verbatim}
\end{enumerate}
\section{Create the RAILS Docker Environment}
\label{sec:railsdockerenv}
The \gls{rails} \glspl{spa} are implemented as Docker containers that require an environment to run in. To create that environment, the following steps are taken:
\begin{enumerate}
    \item Create a Docker volume for the Rails database.
    \begin{verbatim}
    docker volume create --name myRailsDb  
    \end{verbatim}
    \item Create a Docker volume for the Rails images.
    \begin{verbatim}
    docker volume create --name myRailsImages
    \end{verbatim}
    \item Create a Docker network for the Rails containers to communicate over.
    \begin{verbatim}
    docker network create myRailsNet
    \end{verbatim}
\end{enumerate}
\section{Run the RAILS Docker Containers}
\label{sec:railsdockercontainers}
The Docker containers are pulled from Docker Hub and run on the host machine. The following steps are taken to run the \gls{rails} Docker containers:
\begin{enumerate}
    \item Run a Docker container for the MongoDB database.
        \begin{verbatim}
        docker run --network myRailsNet --name myMongo -v myRailsDb:/data/db
        -p 27017:27017 -d mongo
        \end{verbatim}
    \item Run a Docker container for the \gls{rids} \gls{ds}.
        \begin{verbatim}
        docker run --network myRailsNet --name myRids -p 3000:3000 -d dbristow/rids
        \end{verbatim}
    \item Run a Docker container for the \gls{rlds} \gls{ds}.
        \begin{verbatim}
        docker run --network myRailsNet --name myRlds -p 3006:3006 -d dbristow/rlds
        \end{verbatim}
    \item Run a Docker container for the Mosquitto broker.
        \begin{verbatim}
        docker run --network myRailsNet -it --name myMqttBrkr -p 1883:1883 -p 9001:9001
        -v mosquitto:/mosquitto -d eclipse-mosquitto
        \end{verbatim}
    \item Run a Docker conatainer for the \gls{isrs} \gls{iots}.
        \begin{verbatim}
        docker run --network myRailsNet -p 3005:3005 --name myIsrs -d dbristow/isrs
        \end{verbatim}
    \item Run a Docker conatainer for the \gls{isms} \gls{iots}.
        \begin{verbatim}
        docker run --network myRailsNet --name myIsms -d dbristow/isms
        \end{verbatim}
    \item Run a Docker container for the \gls{rsrm} \gls{spa}.
        \begin{verbatim}
        docker run --network myRailsNet -p 3002:8080 --name myRsrm -d dbristow/rsrm
        \end{verbatim}
\end{enumerate}

To bring up the  \gls{rsrm} \gls{spa} point the browser to \href{http://localhost:3002}{http://localhost:3002}\\
\\
To stop the \gls{rails} Docker containers, the following command is used:
\begin{verbatim}  
    docker stop myRsrm myIsms myIsrs myRids myRlds myMongo myMqttBrkr
\end{verbatim}

To restart the \gls{rails} Docker containers, the following command is used:
\begin{verbatim}
    docker start myMongo myMqttBrkr myIsms myIsrs myRids myRsrm
\end{verbatim}

To remove the \gls{rails} Docker containers, the following commands are used:
\begin{verbatim}
    docker rm myRsrm myIsms myIsrs myRids myRlds myMongo myMqttBrkr
\end{verbatim}

