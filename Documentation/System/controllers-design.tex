\chapter{Controllers Design}
\label{chap:controllers-design}
\section{Introduction}
\label{sec:introduction}
Figure \ref{fig:microarchitecture} shows the microservices components that make up the design of \gls{rails}. The microservices,
highlighted with the light green colored background in Figure \ref{fig:microarchitecture} are shown connected to the \gls{iot} components on the left side of the figure.
The \gls{iot} components are the controllers that are used to control the model railroad layout. The controllers are connected to the \gls{mqtt} broker via \gls{wifi}. 
The controllers are programmed to provide model railroad sensors and actuators. A conceptual diagram of the controller design is shown in Figure 
\ref{fig:system-concept} where the \gls{pc} is the platform the \gls{mqtt} broker and the other microservices are running.
\begin{figure}[htbp]
    \centering
    \includegraphics[scale=0.2]{../Images/system-concept.png}
    \caption{Components in the design of RAILS}
    \label{fig:system-concept}
\end{figure}

\section{Microcontrollers}
\label{sec:controllers}
A NodeMCU, an ESP8266 development board is a small, breadboard-friendly circuit board that houses a \gls{wifi} microcontroller chip along with additional components to make 
it easy to use for programming and prototyping. It's a popular choice for hobbyists and makers working on \gls{iot} projects.
The key features of the NodeMCU are:
\begin{itemize}
	\item 32-bit \gls{risc} \gls{cpu}: Tensilica Xtensa LX106 running at 80 MHz
	\item 64 KiB of instruction \gls{ram}, 96 KiB of data \gls{ram}
	\item \gls{ieee} 802.11 b/g/n \gls{wifi}
	\item 16 \gls{gpio} pins
	\item \gls{spi}, \gls{i2c}, \gls{i2s} interfaces with \gls{dma} (sharing pins with \gls{gpio})
	\item \gls{uart} on dedicated pins
	\item 1 10-bit \gls{adc}
	\item \gls{pwm} on all \gls{gpio} pins with \gls{dma} (sharing pins with \gls{gpio})
	\item 1 8-bit \gls{dac}
	\item 10 µA deep sleep current
\end{itemize}
The benefits of using an ESP8266 development board, such as the NodeMCU are:
\begin{itemize}
	\item Breadboard-friendly
	\item Versatile and easy to use
	\item ESP8266 boards are some of the cheapest \gls{wifi} development boards available
	\item Easy to program and debug, either with the Arduino \gls{ide} or \gls{vscode} with PlatformIO.
	\item Easy to connect to a network and the Internet
\end{itemize}

\section{RFID Controller} 
\label{sec:rfid-controller}
The \gls{rfid} controller is responsible for reading the \gls{rfid} tags attached to the rolling stock. The \gls{rfid} controller is programmed to:
\begin{itemize}
	\item initializes \gls{wifi} and \gls{mqtt} paramters are set at compile time using values kept in a params.h file
	\item connects to an \gls{mqtt} broker via \gls{wifi}
	\item publishes info about this reader to the topic ``micros'' in the format: \\
\{``et'':``1590462747'',``mcntrlr'':``RfidRdr01'',``msgType'':``initial'',``ip'':``192.168.0.19''\}
	\item publishes a heartbeat to the topic ``micros'' in the format: \\
\{``et'':``1590462747'',``mcntrlr'':``RfidRdr01'',``msgType'':``heartbeat''\}
	\item reads values from a single ID-12LA \gls{rfid} reader, formats the results as a \gls{json} string, 
gets Epoch time from an \gls{ntp} server and then publishes the \gls{json} string to the topic ``sensors/rfid''
in the format: \\
\{``et'':``1590463450'',``mcntrlr'':``RfidRdr01'',``reader'':``1'',``rfid'':``1C0044CF23''\}
\end{itemize}

Figure \ref{fig:rfid_schematic} depicts the circuit diagram of the \gls{rfid} controller. The \gls{rfid} reader is connected to the ESP8266 microcontroller via a serial connection.

\begin{figure}[htbp]
    \centering
    \includegraphics[scale=0.18]{../Images/rfid_schematic.png}
    \caption{RFID Controller}
    \label{fig:rfid_schematic}
\end{figure}

\section{Turnout Controller}
\label{sec:turnout-controller}
The turnout controller is responsible for controlling the turnouts on the model railroad layout. The turnout controller is programmed to:
\begin{itemize}
    \item initializes \gls{wifi} and \gls{mqtt} paramters are set at compile time using values kept in a params.h file
    \item connects to an \gls{mqtt} broker via \gls{wifi}
    \item publishes info about this reader to the topic ``micros'' in the format: \\
    \{``et'':``1590462747'',``mcntrlr'':``TrnCntlr01'',``msgType'':``initial'',``ip'':``192.168.0.19''\}
    \item publishes a heartbeat to the topic ``micros'' in the format: \\
    \{``et'':``1590462747'',``mcntrlr'':``TrnCntlr01'',``msgType'':``heartbeat''\}
    \item subscribes to the topic acts/to/TrnCntlrxx where xx is the this controller in the format: \\
    \{``to'':``1|2|3|4'',``command'':``THROW|CLOSE|STATUS''\} which either sets the turnout using an L293 to closed or thrown based on the command received
    \item responds to status command for a turnout by publishing the status of the turnout to the topic ''sensors/toc'' in the format: \\
    \{``et'':``1590463450'',``mcntrlr'':``TrnCntlr01'',``to'':``1'',``dir'':``THROWN''\}
\end{itemize}

Figure \ref{fig:turnout-controller} depicts the circuit diagram of the turnout controller.

\begin{figure}[htbp]
    \centering
    \includegraphics[scale=0.4]{../Images/turnout_schematic.png}
    \caption{Turnout Controller}
    \label{fig:turnout-controller}
\end{figure}

\section{LED Push Button Controller}
\label{sec:lcb-controller}
The \gls{led} push button controller is responsible to display the direction of an associated turnout and allow a user to switch the direction of that turnout. The \gls{led} push button controller is programmed to:
\begin{itemize}
    \item Initialization of the NodeMCU and configuration of the GPIO pins.
    \item Initialization of the \gls{wifi} connectivity and connect to the \gls{mqtt} broker. This allows the controller to send and receive messages over the network. At the finish of the initialization process, the controller will publish a message to the topic ``micros''  indicating its readiness and status to the \gls{mqtt} broker. The message is in \gls{json} format as follows:\\
\{``et'':``1590462747'':``mcntrlr'':``LpbCntlr01'':``msgType'':``initial'':``ip'':``192.168.0.19''\}
    \item Subscribes to the topic acts/tpl/LpbCntlrxx where xx is a specific controller from the \gls{mqtt} broker. The message is in \gls{json} format as follows:\\
\{``cntrlr'':``LpbCntlrxx'':``lamp'':``1|2|3...6'':``color'':``RED|GREEN|BLUE|YELLOW''\}
    \item Reads up to 6 buttons (count set in ``params.h'') using simple debouncing and publishes a \gls{json} message on press to the topic ``sensors/pb'' to the \gls{mqtt} broker. The message is in \gls{json} format as follows:\\
    \{``et'':``1590462747'':``mcntrlr'':``LpbCntlrxx'':``pb'':``1|2|3...6''\}
    \item Locks that button until a command arrives on the command topic with a matching ``lamp'' number is recieved.
    \item Sending periodic heartbeat messages to the \gls{mqtt} broker to indicate that the controller is operational. This can be used for monitoring and debugging purposes. The message are in \gls{json} format as follows:\\
\{``et'':``1590462747'',``mcntrlr'':``LpbCntlrxx'',``msgType'':``heartbeat''\}
\end{itemize}

Figure \ref{fig:lpb_schematic} depicts the circuit diagram of the \gls{led} push button controller.

\begin{figure}[H]
  \centering
    \includegraphics[scale=0.14]{../Images/lpb_schematic.png}
  \caption{LED Push Button Controller}
  \label{fig:lpb_schematic}
\end{figure}
