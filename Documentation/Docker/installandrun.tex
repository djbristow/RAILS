\chapter{Setup and Run}
\label{ch:installandrun}
\section{Docker Installation}
\label{sec:dockerinstall}
The \gls{rails} \glspl{spa} are implemented as Docker containers. The Docker containers are built and pushed to Docker Hub. The Docker containers are then pulled from Docker Hub and run on the host machine. 
The host machine must have Docker installed. Docker is a platform for developing, shipping, and running applications in containers. Docker can be installed on Windows, macOS, and Linux. 
The installation instructions for Docker can be found at \href{https://docs.docker.com/get-docker/}{https://docs.docker.com/get-docker/}.
\section{Docker Compose Installation}
\label{sec:dockercomposeinstall}
Docker Compose is a tool for defining and running multi-container Docker applications. Docker Compose uses a YAML file to configure the application's services. The installation instructions\footnote{Note that for Microsoft Windows the installation of the \gls{gui} Docker Desktop automatically installs Docker Compose.} for Docker Compose can be found at \href{https://docs.docker.com/compose/install/}{https://docs.docker.com/compose/install/}.  The Docker Compose file for the \gls{rails} \glspl{spa} is shown in Appendix \ref{app:yamlfile}.
\section{Broker Configuration}
\label{sec:brokerconfig}
The \gls{rails} \glspl{spa} use the Mosquitto broker to communicate with each other. The Mosquitto broker is an open-source message broker that implements the \gls{mqtt} protocol. The Mosquitto broker must be configured to allow the \gls{rails} \glspl{spa} to communicate with each other. The Mosquitto broker configuration file must be edited to allow the \gls{rails} \glspl{spa} to communicate with each other. The configuration file is found in the virtual storage whose volume name is “mosquitto”. To establish and modify the configuration file the following steps are taken:
\begin{enumerate}
    \item Create a Docker volume for the Mosquitto broker.
    \begin{verbatim}
    docker volume create --name mosquitto
    \end{verbatim}
    \item Run a Docker container for the first time the Mosquitto broker, which will create the initial configuration file.
    \begin{verbatim}
    docker run -it --name myMqttBrkr -p 1883:1883 -p 9001:9001 --rm
    -v mosquitto:/mosquitto -d eclipse-mosquitto
    \end{verbatim}
    \item Stop the broker container.
    \begin{verbatim}    
    docker stop myMqttBrkr
    \end{verbatim}
    \item Locate the configuration file in the "mosquitto" volume.
    \begin{verbatim}    
    docker inspect mosquitto
    \end{verbatim}
    \item Edit the configuration file modifying or adding the following lines:
    \begin{verbatim}
    listener 1883
    allow_anonymous true
    socket_domain ipv4
    \end{verbatim}
\end{enumerate}
\section{Create the RAILS Docker Environment}
\label{sec:railsdockerenv}
The \gls{rails} \glspl{spa} are implemented as Docker containers that require an environment to run in. To create that environment, the following steps are taken:
\begin{enumerate}
    \item Create a Docker volume for the Rails database.
    \begin{verbatim}
    docker volume create --name myRailsDb  
    \end{verbatim}
    \item Create a Docker volume for the Rails images.
    \begin{verbatim}
    docker volume create --name myRailsImages
    \end{verbatim}
    \item Create a Docker network for the Rails containers to communicate over.
    \begin{verbatim}
    docker network create myRailsNet
    \end{verbatim}
\end{enumerate}
\section{Run the RAILS Docker Containers}
\label{sec:railsdockercontainers}
The Docker containers are pulled from Docker Hub and run on the host machine. To run all of the \gls{rails} Docker containers, either create the YAML file from Appendix \ref{app:yamlfile} or retrieve the YAML file from the \gls{rails} GitHub repository found at \href{https://github.com/djbristow/RAILS/tree/master/Docker%20Based}{my GitHub repository}. The YAML file is used to run the \gls{rails} Docker containers.  
to run all four \gls{rails} Docker containers, the following command in the directory where the YAML file is located:
\begin{verbatim}
docker-compose up -d
\end{verbatim}
If the \gls{rails} Docker \gls{spa} containers are to be run individually, the following commands are used:
\begin{verbatim}
docker-compose up -d myMrim
docker-compose up -d myRsrm
docker-compose up -d myMppm
docker-compose up -d myMrlm
\end{verbatim}
Point the browser to the following URLs to access the \gls{rails} \glspl{spa}:
\begin{itemize}
    \item \gls{mrim} \gls{spa} - \href{http://localhost:3001}{http://localhost:3001}
    \item \gls{rsrm} \gls{spa} - \href{http://localhost:3002}{http://localhost:3002}
    \item \gls{mppm} \gls{spa} - \href{http://localhost:3008}{http://localhost:3008}
    \item \gls{mrlm} \gls{spa} - \href{http://localhost:3004}{http://localhost:3004}
\end{itemize}
To stop and remove the \gls{rails} Docker containers, the following command is used:
\begin{verbatim}
docker-compose down
\end{verbatim}

See Appendix \ref{app:examplecommands} for screenshots of the Docker commands used to create the Docker environment and run the \gls{rails} \glspl{spa}.